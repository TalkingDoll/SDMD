\documentclass{beamer}
\usefonttheme{serif}
\usepackage{bm}
\usepackage{amsmath, amssymb}
\usepackage{mathtools}

\usetheme{Madrid}
\usecolortheme{whale}

\title{Data-Driven Methods for Dynamical Systems}
\author{Yuanchao Xu}
\institute{University of Alberta}
\date{\today}

\begin{document}
	
	\begin{frame}
		\titlepage
	\end{frame}
	
	\begin{frame}{Outline}
		\tableofcontents
	\end{frame}
	
	\section{Introduction}
	
	\begin{frame}{Introduction}
		\begin{itemize}
			\item Complex dynamical systems challenge:
			\begin{itemize}
				\item Understanding and predicting long-term behavior
				\item Traditional models often inadequate for complex systems
			\end{itemize}
			\item Data-driven methods advantages:
			\begin{itemize}
				\item Extract insights directly from experimental/simulation data
				\item Reduce dependence on explicit mathematical models
				\item Capture hidden dynamics not easily modeled analytically
			\end{itemize}
			\item Koopman operator theory benefits:
			\begin{itemize}
				\item Transforms nonlinear dynamics into linear representations
				\item Enables powerful spectral analysis tools for nonlinear systems
				\item Bridges data-driven approaches with dynamical systems theory
			\end{itemize}
		\end{itemize}
	\end{frame}
	
	\section{Koopman Operator Theory}
	\begin{frame}{Koopman Operator Theory - Overview}
		\begin{itemize}
			\item Introduced by B.O. Koopman in 1931
			\item Core idea: Lift nonlinear dynamics to linear but infinite-dimensional space
			\item Consider a dynamical system $(\mathcal{M}, \mu)$: 
			$$\dot{x} = f(x)$$
			where $x \in \mathcal{M} \subseteq \mathbb{R}^d$ and $\mu$ is a probability measure.
			\item Koopman operator $\mathcal{K}: L^2(\Omega, \mu) \to L^2(\Omega, \mu) $ acts on observable $g: \mathcal{M} \to \mathbb{C}$:
			\begin{equation*}
				(\mathcal{K}g)(x) = g(f(x))
			\end{equation*}
			\item Key properties:
			\begin{itemize}
				\item Linear: $\mathcal{K}(\alpha g_1 + \beta g_2) = \alpha \mathcal{K}g_1 + \beta \mathcal{K}g_2$
				\item Infinite-dimensional: Operates on function space
				\item Preserves nonlinear dynamics information
			\end{itemize}
		\end{itemize}
	\end{frame}
	
	\begin{frame}{Koopman Operator Theory: Methods and Applications}
		\begin{itemize}
			\item Data-driven approximation methods:
			\begin{itemize}
				\item Extended Dynamic Mode Decomposition (EDMD)
				\item Residual Dynamic Mode Decomposition (ResDMD)
				\item Generator Extended Dynamic Mode Decomposition (gEDMD)
			\end{itemize}
			\item Key applications:
			\begin{itemize}
				\item Model reduction for complex systems
				\item Identification of coherent structures in fluid dynamics
				\item Stability analysis
				\item Nonlinear control problems
			\end{itemize}
		\end{itemize}
	\end{frame}
	
	
	\section{Koopman Operator Computation for Deterministic Systems}
	
	\begin{frame}{Extended Dynamic Mode Decomposition}
		\begin{itemize}
			\item Collect Data: Gather i.i.d. data points \( \{ x_1, \dots, x_m \} \) and their corresponding next states \( \{ y_1, \dots, y_m \} \).
			
			\item Construct the Data Matrices:
			Define the data matrices \( X \) and \( Y \):
			\begin{gather*}
				X = \begin{bmatrix} 
					\underset{\rule{0.1mm}{5mm}}{\overset{\rule{0.1mm}{5mm}}{x_1}} & \cdots & \underset{\rule{0.1mm}{5mm}}{\overset{\rule{0.1mm}{5mm}}{x_m}}
				\end{bmatrix}, \
				Y = \begin{bmatrix} 
					\underset{\rule{0.1mm}{5mm}}{\overset{\rule{0.1mm}{5mm}}{y_1}} & \cdots & \underset{\rule{0.1mm}{5mm}}{\overset{\rule{0.1mm}{5mm}}{y_m}}
				\end{bmatrix}
			\end{gather*}
			Evaluate the dictionary \( \Psi = \{\psi_1, \dots, \psi_N\} \) on these data to form matrices:
			\begin{gather*}
				\Psi_X =\begin{bmatrix}
					\psi_1(x_1) & \cdots & \psi_{N}(x_1) \\
					\vdots & \ddots & \vdots \\
					\psi_1(x_m) & \cdots & \psi_{N}(x_m)
				\end{bmatrix} , \
				\Psi_Y = \begin{bmatrix}
					\psi_1(y_1) & \cdots & \psi_{N}(y_1) \\
					\vdots & \ddots & \vdots \\
					\psi_1(y_m) & \cdots & \psi_{N}(y_m)
				\end{bmatrix}
			\end{gather*}
		\end{itemize}
	\end{frame}
	
	\begin{frame}{Extended Dynamic Mode Decomposition}
		\begin{itemize}
			\item Solve the Linear System:
			Solve the least-squares problem to find \( \mathbf{K} \):
			\[
			\mathbf{K} = \Psi_X^\dagger \Psi_Y = \widehat{G}^\dagger \widehat{A}
			\]
			where $\widehat{G}\coloneqq\frac{1}{m}\Psi_X^* \Psi_X, \widehat{A}\coloneqq\frac{1}{m}\Psi_X^* \Psi_Y$, \( \dagger \) is the pseudoinverse.
			
			\item Koopman Modes and Eigenvalues:
			Once \( \mathbf{K} \) is computed, solve the eigenvalue problem:
			\[
			\mathbf{K} \mathbf{v} = \lambda \mathbf{v}
			\]
			where \( \lambda \) are the Koopman eigenvalues, and \( \mathbf{v} \) are the (right) eigenvectors.   
		\end{itemize}
	\end{frame}
	
	\begin{frame}{Extended Dynamic Mode Decomposition}
		\begin{itemize}
			\item Eigenvalues $\lambda_j$: frequencies or growth rates of the system's dynamics
			\item Eigenfunctions $\phi_j$: Functions satisfying $\mathcal{K}\phi_j = \lambda_j \phi_j$
			\begin{itemize}
				\item Computed as: $\phi_j(\mathbf{x}) = \sum_{k=1}^N \mathbf{v}_{jk} \psi_k(\mathbf{x})$
				where $\mathbf{v}_j = [v_{j1}, \ldots, v_{jN}]^T$ is the $j$-th right eigenvector of $\mathbf{K}$
				\item Capture fundamental patterns in the nonlinear dynamics
			\end{itemize}
			\item Koopman Modes $\bm{\xi}_j$: Spatial patterns associated with eigenfunction $\phi_j$
			\begin{itemize}
				\item For observable $\mathbf{g} = [g_1(x), g_2(x), \dots, g_n(x)]^T$:
				\[
				\bm{\xi}_j = \left[ \frac{\langle g_1, \phi_j \rangle_\mu}{\langle \phi_j, \phi_j \rangle_\mu}, \frac{\langle g_2, \phi_j \rangle_\mu}{\langle \phi_j, \phi_j \rangle_\mu}, \dots, \frac{\langle g_n, \phi_j \rangle_\mu}{\langle \phi_j, \phi_j \rangle_\mu} \right]^T
				\]
			\end{itemize}
			\item Observable decomposition and prediction:
			\begin{itemize}
				\item At time 0: $\mathbf{g}(\mathbf{x}_0) \approx \sum_{j=1}^N \phi_j(\mathbf{x}_0)\bm{\xi}_j$
				\item After applying $\mathcal{K}$ for $n$ times: $\mathbf{g}(\mathbf{x}_n) = \mathcal{K}^n\mathbf{g}(\mathbf{x}_0) \approx \sum_{j=1}^N \lambda_j^n \phi_j(\mathbf{x}_0)\bm{\xi}_j $
			\end{itemize}
		\end{itemize}
	\end{frame}
	
	\begin{frame}{Convergence in large data limit}
		For finite $m$ data points, the $ij$-th element of $\widehat{G}$ and $\widehat{A}$ are:
		\begin{gather*}
			[\widehat{G}]_{ij} = \frac{1}{m}\sum_{i=1}^m \overline{\psi}_i(x_i)\psi_j(x_i) \\
			[\widehat{A}]_{ij} = \frac{1}{m}\sum_{i=1}^m \overline{\psi}_i(x_i)\psi_j(y_i)
		\end{gather*}
		In the large data limit, EDMD converges to a Galerkin projection, i.e., as $m\to\infty$, by SLLN we have 
		\begin{align*}
			\lim_{m\to\infty}[\widehat{G}]_{ij} &\to \langle \psi_i, \psi_j \rangle_{\mu} \\
			\lim_{m\to\infty}[\widehat{A}]_{ij} &\to \langle \psi_i, \mathcal{K}\psi_j \rangle_{\mu}
		\end{align*}
		
	\end{frame}
	
	\begin{frame}{Residual Dynamic Mode Decomposition (ResDMD) (1/3)}
		\begin{itemize}
			\item Addresses spectral pollution in EDMD
			\item Considers both $\mathcal{K}$ and $\mathcal{K}^*\mathcal{K}$
			\item Computes squared relative residual:
			\[ \text{res}(\lambda, g)^2 \coloneqq \frac{\int_\Omega |\mathcal{K}g(x) - \lambda g(x)|^2 d\mu(x)}{\int_\Omega |g(x)|^2 d\mu(x)} \]
		\end{itemize}
	\end{frame}
	
	\begin{frame}{Residual Dynamic Mode Decomposition (ResDMD) (2/3)}
		\begin{itemize}
			\item For normalized eigenfunction $g$, this becomes:
			\begin{align*}
				\text{res}(\lambda, g)^2 = \sum_{i,j=1}^{N_K} \overline{v}_i & \left( \langle \mathcal{K}\psi_i, \mathcal{K}\psi_j \rangle_\mu - \lambda \langle \psi_i, \mathcal{K}\psi_j \rangle_\mu \right. \\
				& \left. - \overline{\lambda} \langle \mathcal{K}\psi_i, \psi_j \rangle_\mu + |\lambda|^2 \langle \psi_i, \psi_j \rangle_\mu \right) v_j
			\end{align*}
			\item Where $g = \mathbf{\Psi} v$ for some $v \in \mathbb{C}^N$
		\end{itemize}
	\end{frame}
	
	\begin{frame}{Residual Dynamic Mode Decomposition (ResDMD) (3/3)}
		\begin{itemize}
			\item Practical computation of residual using $\Psi_X$ and $\Psi_Y$:
			\begin{align*}
				\widehat{\text{res}}(\lambda, g)^2 := \frac{1}{m} v^* & \left( \Psi_Y^* \Psi_Y - \lambda (\Psi_X^* \Psi_Y)^* \right. \\
				& \left. - \overline{\lambda} \Psi_X^* \Psi_Y + |\lambda|^2 \Psi_X^* \Psi_X \right) v
			\end{align*}
			\item Provides more accurate spectral approximation
			\item Allows detection and discarding of spurious eigenvalues
		\end{itemize}
	\end{frame}
	
	\begin{frame}{Neural Network ResDMD (NN-ResDMD)}
		\begin{itemize}
			\item Uses neural network to learn dictionary functions
			\item Adaptive to complex systems
			\item Loss function: $J_K = \frac{1}{\sqrt{m}} \|\Psi_Y - \Psi_X KV\|_F^2$
			\item Iterative process:
			\begin{itemize}
				\item Update $K = (G + \sigma I)^\dagger A$
				\item Adjust network parameters using gradient descent
			\end{itemize}
		\end{itemize}
	\end{frame}
	
	\section{Koopman Operator Computation for Stochastic Systems}
	
	\begin{frame}{EDMD for Koopman Generator (gEDMD)}
		\begin{itemize}
			\item Approximates generator $A$ for continuous-time stochastic systems
			\item SDE: $dX_t = b(X_t)dt + \sigma(X_t)dW_t$
			\item Generator: $Ag(x) = b(x) \cdot \nabla g(x) + \frac{1}{2}\sum_{i,j} a_{ij}(x) \frac{\partial^2g(x)}{\partial x_i \partial x_j}$
			\item Approximation: $A \approx \Psi_X^\dagger \dot{\Psi}_X$
		\end{itemize}
	\end{frame}
	
	\begin{frame}{Stochastic Extended DMD (S-EDMD)}
		\begin{itemize}
			\item Addresses unboundedness of generator in stochastic systems
			\item Constructs sequence of bounded operators $A_n$
			\item $A_n := \frac{K_n(\frac{1}{n}) - I}{\frac{1}{n}}$
			\item Uses Trotter-Kato Approximation theorem for convergence
			\item Ensures numerical stability and accuracy
		\end{itemize}
	\end{frame}
	
	\section{Conclusion}
	
	\begin{frame}{Conclusion}
		\begin{itemize}
			\item Addressed limitations of existing Koopman operator approximation methods
			\item Proposed NN-ResDMD for complex deterministic systems
			\item Introduced S-EDMD for stochastic systems
			\item Future work:
			\begin{itemize}
				\item Implementation and testing
				\item Performance evaluation across various dynamical systems
			\end{itemize}
		\end{itemize}
	\end{frame}
	
	\begin{frame}{Thank You}
		\begin{center}
			\Large{Thank you for your attention!}
			
			\vspace{1cm}
			
			Any questions?
		\end{center}
	\end{frame}
	
\end{document}